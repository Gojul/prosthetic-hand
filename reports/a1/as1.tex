%%
%% Copyright 2007, 2008, 2009 Elsevier Ltd
%%
%% This file is part of the 'Elsarticle Bundle'.
%% ---------------------------------------------
%%
%% It may be distributed under the conditions of the LaTeX Project Public
%% License, either version 1.2 of this license or (at your option) any
%% later version.  The latest version of this license is in
%%    http://www.latex-project.org/lppl.txt
%% and version 1.2 or later is part of all distributions of LaTeX
%% version 1999/12/01 or later.
%%
%% The list of all files belonging to the 'Elsarticle Bundle' is
%% given in the file `manifest.txt'.
%%

%% Template article for Elsevier's document class `elsarticle'
%% with numbered style bibliographic references
%% SP 2008/03/01
%%
%%
%%
%% $Id: elsarticle-template-num.tex 4 2009-10-24 08:22:58Z rishi $
%%
%%
\documentclass[12pt,3p]{elsarticle}
\makeatletter
\def\ps@pprintTitle{%
 \let\@oddhead\@empty
 \let\@evenhead\@empty
 \def\@oddfoot{\centerline{\thepage}}%
 \let\@evenfoot\@oddfoot}
\makeatother

%% Use the option review to obtain double line spacing
%% \documentclass[preprint,review,12pt]{elsarticle}

%% Use the options 1p,twocolumn; 3p; 3p,twocolumn; 5p; or 5p,twocolumn
%% for a journal layout:
%% \documentclass[final,1p,times]{elsarticle}
%% \documentclass[final,1p,times,twocolumn]{elsarticle}
%% \documentclass[final,3p,times]{elsarticle}
%% \documentclass[final,3p,times,twocolumn]{elsarticle}
%% \documentclass[final,5p,times]{elsarticle}
%% \documentclass[final,5p,times,twocolumn]{elsarticle}

%% if you use PostScript figures in your article
%% use the graphics package for simple commands
%% \usepackage{graphics}
%% or use the graphicx package for more complicated commands
%% \usepackage{graphicx}
%% or use the epsfig package if you prefer to use the old commands
%% \usepackage{epsfig}

%% The amssymb package provides various useful mathematical symbols
\usepackage{amssymb}
\usepackage{amsmath}

%% The lineno packages adds line numbers. Start line numbering with
%% \begin{linenumbers}, end it with \end{linenumbers}. Or switch it on
%% for the whole article with \linenumbers after \end{frontmatter}.
%% \usepackage{lineno}

%% natbib.sty is loaded by default. However, natbib options can be
%% provided with \biboptions{...} command. Following options are
%% valid:

%%   round  -  round parentheses are used (default)
%%   square -  square brackets are used   [option]
%%   curly  -  curly braces are used      {option}
%%   angle  -  angle brackets are used    <option>
%%   semicolon  -  multiple citations separated by semi-colon
%%   colon  - same as semicolon, an earlier confusion
%%   comma  -  separated by comma
%%   numbers-  selects numerical citations
%%   super  -  numerical citations as superscripts
%%   sort   -  sorts multiple citations according to order in ref. list
%%   sort&compress   -  like sort, but also compresses numerical citations
%%   compress - compresses without sorting
%%
%% \biboptions{comma,round}

% \biboptions{}

\usepackage{amsmath}

\journal{}

\begin{document}

\begin{frontmatter}

\title{MCEN30019 Assignment 1 \\
\small Prosthetic Hand Development for Landmine Victims \\
\small Summary of Design Objectives and Criteria}

\author{Lucas Brouwer}
\author{George Juliff}
\author{Adam Kues}
\author{Thomas Miles}

\begin{abstract}
Current on-market hand prostheses are both expensive and often lack the degree of control needed to perform many daily activities. This project involves designing and prototyping an affordable prosthetic hand for e-NABLE, a charity currently distributing prostheses to Colombian landmine victims. The client \citep{Walbran} has requested the prosthesis be designed for a 50 year old man, who has undergone transradial amputation as a result of his injuries. This report summarises the design objectives and criteria and proposes an objective function for assessing the quality of future designs.
\end{abstract}

\end{frontmatter}

%%
%% Start line numbering here if you want
%%
% \linenumbers

%% main text
\section{Design Objectives}
\label{sec1}
\begin{itemize}
\item $Obj_1$: Allows user to act bimanually, with more dexterity than competing alternatives. \\\\
To be useful to the charity the hand must be more advanced than low technology
solutions already available on the market (eg. a hook). One conducted survey shows that over 70\% of users
want a level of control that allows them to operate electronic and manual devices,
which often requires control of single digits~\cite{Pylatiuk}. In addition, a
user study by Peerdeman et al. shows a large proportion of users want to be able to 
carry and manipulate objects with both hands, so
a solution that allows for bimanual control is optimal~\cite{Peerdeman}.
\item $Obj_2$: Cost and labour effective to produce and maintain. \\\\
The prosthetic is being produced by a charity, meaning it must not require expensive
specialized equipment to create or assemble parts and the cost of materials must
be cheap. Estimations suggest that in low-income countries like Colombia over 80\%
of people cannot afford prosthetic care~\cite{Slade}; so a low cost solution is vital to make any
sort of impact. In addition, the e-NABLE charity currently develops a prosthetic with
a two week turnaround, so to improve the status quo a fast manufacture time is
desired~\cite{Walbran}.\pagebreak
\item $Obj_3$: Does not deter a user from wearing it. \\\\
The prosthetic must be unobtrusive in order to allow the user to perform their
daily tasks. An overly obtrusive or heavy hand is a common concern; a study by
Pylatiuk et al. mentions that some users find the hand requires too much attention,
and that they would prefer a prosthetic that is self-supporting~\cite{Pylatiuk}.

\end{itemize}

%%///////////////////////////////////////////George's addition//////////////////////////////////////////////////////////%

\section{Essential Criteria}
\label{sec2}
\begin{itemize}
\item $E_1$: The prosthetic should be comfortable enough for extended use. ($Obj_3$)
	\begin{itemize}
	%\item {\bf CONTRIBUTIONS}: contributes primarily to $Obj_3$.
	
	\item $E_{1A}$: Portion attached to forearm must not weigh over $M_{Wmax}$ = 500g.
		\begin{itemize}
		\item Justification: As recommended in a review of current prosthesis~\cite{Belter} the weight of a prosthesis for a a wrist disarticulation should be below this threshold. However, as found in a compilation of surveys~\cite{Cordella} even at or below this threshold weight is often a major concern of the user and as such using this recommendation for our transradial prosthesis will help address this issue.
		\end{itemize}
		
	\item $E_{1B}$: Unit must not cause pain, irritation or significant discomfort during the course of standard operation, in line with similar products.
		\begin{itemize}
		\item Justification: This is necessary for daily operation but due to the number of variables and metrics, in addition to individual tolerances that determine this, can only practically be determined through experimentation on potential mounting options.
		\end{itemize}
		
	\item $E_{1C}$: Forearm section should not exceed 1.1 times the natural width of forearm.
		\begin{itemize}
		\item Justification: This was determined by our team through experimentation to be sufficient to not significantly impede clothing choice, specifically dress shirts.
		\end{itemize}
		
	\end{itemize}
	
\item $E_2$: The prosthetic should be time and cost effective to produce and maintain. ($Obj_2$)
	\begin{itemize}
	%\item {\bf CONTRIBUTIONS}: contributes primarily to $Obj_2$.
	\item $E_{2A}$: Unit must cost less than \$500AU.
		\begin{itemize}
		\item Justification: This is an explicit requirement of the client~\cite{Walbran}.
		\end{itemize}
		
	\item $E_{2B}$: Unit should have under a 2-week turnaround.
		\begin{itemize}
		\item Justification: This is required to match e-NABLE’s current standard~\cite{Armfield}.
		\end{itemize}
		
	\item $E_{2C}$: Repair for minor/common failures/damages require at most 1 man hour and 6 hours.
		\begin{itemize}
		\item Justification: this is essential to make same day repairs on these issues feasible, which is necessary to allow users to rely on the prosthesis as according to a survey~\cite{Pylatiuk} most adults use their prosthesis for upwards of 8 hours daily.
		\end{itemize}
		
	\end{itemize}
	
\item $E_3$: The prosthetic should be able to assist the user in daily tasks. ($Obj_1$, $Obj_3$)
	\begin{itemize}
	%\item {\bf CONTRIBUTIONS}: contributes to $Obj_1$ and $Obj_3$.
	\item $E_{3A}$: Unit must have a palmer pinch force of at least $M_{Fmin}$ = 65N.
	\item $E_{3B}$: Unit must have at a closing speed of at least $M_{Smin}$ = 115 $^{\circ}$/s
		\begin{itemize}
		\item Justification: Both 3A and 3B are recommended minimums of report~\cite{Belter} to facilitate daily tasks.
		\end{itemize}
		
	\item $E_{3C}$: Unit must have a battery life of $M_{Bmin}$ = 10 hours under standard operating conditions or allow for easy replacement of batteries.
		\begin{itemize}
		\item Justification: based off the legal working day~\cite{LaE} in Columbia with some leweigh to allow for travel or unforeseen circumstance. In addition the survey~\cite{Pylatiuk} suggests most uses require more than eight hours of use.
		\end{itemize}
		
	\item $E_{3D}$: Gesture classification accuracy of greater than \textit{90\%} for tasks specified in \textit{$E_{3A}$} and \textit{$E_{3B}$}.
	
		\begin{itemize}
		\item Justification: frequent misclassification of user input under normal use will  be severely detrimental to the users ability to act bimanually.
		\end{itemize}
		
	\end{itemize}
	
\item $E_4$: Unit should have a functional lifespan of at least $M_{Lmin}$ = 1 year without major refurbishment or repair. ($Obj_2$, $Obj_3$)
%{\bf CONTRIBUTIONS}: contributes to $Obj_2$ and $Obj_3$.
	\begin{itemize}
	
	\item $E_{4A}$: Unit should be water and dust resistant with an IP rating of at least 54 on exposed sections.
		\begin{itemize}
		\item Justification: This is the minimum standard to survive daily life for a prosthesis that may be rained on and will be exposed to particulates.
		\end{itemize}
		\item $E_{4B}$: the prosthesis should withstand day-to-day use such as lifting household objects, shaking hands, carrying grocery bags, e.t.c. without breaking or putting severe strain on the mechanical components of the device.
		\begin{itemize}
		\item Justification: In order to be reliable throughout the lifespan of the device, all components should stand up to everyday use. Ensuring this will also reduce future maintenance and repair cost.
		\end{itemize}	
	\end{itemize}
	
\end{itemize}

\section{Desirable Criteria}
\label{sec3}
\begin{itemize}
\item The prosthetic should be able to withstand up to a force of $M_{RF_{max}} = 140$N without breaking, nor be particularly susceptible to falling damage or collision damage. ($Obj_1$) \\\\
$D_1 = \text{min}\left\{\dfrac{M_{RF}}{M_{RF_{max}}},1\right\}$ 
	
	\begin{itemize}
	\item Justification: The Bebionic by RSLSteeper is able to withstand a force of 140N; this is the highest force of all the commercially available prosthetic hands~\cite{Cordella}. Fragile or breakable hands are bad for the user,
	especially in countries like Colombia where access to repair or replacement facilities may be limited. 
	\end{itemize}
	\pagebreak
\item The prosthetic should be quiet. ($Obj_3$)\\
$D_2 = \text{max}\{1-\dfrac{M_L}{M_{L_{max}}}, 0\}$
	
	\begin{itemize}
	\item Justification: Over 90\% of the respondents to the survey conducted by~\cite{Pylatiuk} rate the prosthesis sound
	as being at least `a little' distracting, with 10\% rating it `very distracting'. To concentrate properly, SafeWork Australia recommends an ambient noise level of around $M_{L_{max}} = 50$dB as an
	absolute minimum~\cite{Cooper}. Ideally the prosthetic would not make any noise at all, however infeasible this
	may be.
	\end{itemize}
%{\bf CONTRIBUTIONS}: contributes to $Obj_3$.

\item The prosthetic should be dexterous. ($Obj_1$)\\\\
$
	D_3 =  \cfrac{\sum_{n=1}^{n=\text{DOF}} \left(k_n \cdot \text{min}\{\frac{M_{RS}}{M_{RS_{max}}},1\} \right)}{\text{DOF}}$
	\begin{flalign*}
		\text{DOF} &= \text{number of degrees of freedom}&&\\
		k_n &= \text{classification accuracy of DOF n}&&
\end{flalign*}
	
	\begin{itemize}
	\item Justification: The hand should be as precise as possible, to allow the user to grab small objects. The user should also be able to compete in sports, like running and soccer, without trouble. Belter notes that ``\ldots 230$^\circ$/s should be achieved by a high-performing prosthesis''~\cite{Belter}, so we take $M_{RS_{max}} = 230^\circ$/s.\\ The dexterity of each degree of freedom in the hand is weighted by the classification accuracy and averaged.
	\end{itemize}
%{\bf CONTRIBUTIONS}: contributes to $Obj_1$.
\item The prosthetic should be aesthetically pleasing, and should look like a hand as much as possible. ($Obj_3$)\\\\
$D_4 = \cfrac{p_{range}}{100} \times d  \times I \\\\ \text{ Where:} $ 
\begin{flalign*}
d &= 1 - \cfrac{|d_{max} - d_{50}|}{d_{50}}&&\\
d_{max} &= \text{most disproportional measurement for a 50th percentile prosthetic design}\\
d_{50} &= \text{50th percentile measurement corresponding to} d_max\\
I = 1.0 &\rightarrow \text{no external irregularities}\\
0.9 \leq I < 1.0  &\rightarrow \text{minor irregularities, inconspicuous}\\
0.8 \leq I < 0.9 &\rightarrow \text{minor, visible irregularities }\\
0.5 \leq I < 0.8 &\rightarrow \text{major irregularities, inconspicuous}\\
I < 0.5 &\rightarrow \text{major, visible irregularities}\\
p_{range} &=  \text{Percentile range (highest - lowest)}&&\\
\end{flalign*}
\begin{itemize}
	 	
	\item Justification: Irregular protrusions, unnatural colouring, and disproportional components can be unappealing to the user. In the survey conducted in~\cite{Pylatiuk}, approximately 50\% of the respondents indicated they were not entirely happy with the cosmesis of their hand. Using the anthropometry data from \cite{Anthropometric}, the range of custom sizes which can be easily manufactured is given an increased score. Irregularities (e.g. external batteries, unusual colouring, e.t.c.) reduce the aesthetic score using \textit{I} and proportionality discrepancies are accounted for using \textit{d}. 
	\end{itemize}
%{\bf CONTRIBUTIONS}: contributes to $Obj_3$.
\end{itemize}
\section{Objective Function}
\label{sec4} %kjbsdjb
The objective function is: \\
$$D_{overall} = W_1D_1 + W_2D_2 + W_3D_3 + W_4D_4$$ \\
\phantom{aaa}Where $W_1 = 0.15$, $W_2 = 0.25$, $W_3 = 0.4$, $W_4 = 0.2$. \\

Justification: According to the survey conducted by Pylatiuk et al. in ~\cite{Pylatiuk}:
\begin{itemize}
\item There are few concerns regarding the tensile strength of current prosthetics~\cite{Peerdeman}; since $E_{4B}$ must be met for the design to proceed, $W_1$ is weighted the lowest (0.15).
\item A large percentage ($>$90\%) of users find the sound of their hand to be `a little' irritating; $W_2$ is weighted 0.25, since it is a widely reported complaint regarding non-functional behaviour.
\item An equally large percentage ($>$90\%) of users find the grasping speed of current prostheses too slow; furthermore, improved dexterity has the potential to greatly improve the quality of the user experience in day to day life. $W_3$ is weighted 0.4 to reflect this.
\item Around 50\% of users are not entirely happy with the cosmetics of their hand. Since this is a non-functional requirement with fewer users raising complaints than sound, $W_4$ is weighted lower than $W_2$ at 0.2.
\end{itemize}

Given the weightings discussed above, it is expected that trade off's be made between aesthetics ($D_4$) and total tensile strength ($D_1$). This arises due to their similar weightings, and the practical trade off between the bulky, strong finger digits, and more realistic, smaller designs. Audible actuation generally is a result of motors and moving parts; using several small motors to reduce size, or to achieve dexterous, independent finger actuation, and lateral movement,  are likely to make more irritating noise than fewer, larger motors. This gives rise to a trade off between aesthetics ($D_4$), and dexterity ($D_3$) for reduction in noise ($D_2$). 

%% The Appendices part is started with the command \appendix;
%% appendix sections are then done as normal sections
\newpage
\appendix

%% References
%%
%% Following citation commands can be used in the body text:
%% Usage of \cite is as follows:
%%   \cite{key}         ==>>  [#]
%%   \cite[chap. 2]{key} ==>> [#, chap. 2]
%%

%% References with bibTeX database:

\bibliographystyle{elsarticle-num}
% \bibliographystyle{elsarticle-harv}
% \bibliographystyle{elsarticle-num-names}
% \bibliographystyle{model1a-num-names}
% \bibliographystyle{model1b-num-names}
% \bibliographystyle{model1c-num-names}
% \bibliographystyle{model1-num-names}
% \bibliographystyle{model2-names}
% \bibliographystyle{model3a-num-names}
% \bibliographystyle{model3-num-names}
% \bibliographystyle{model4-names}
% \bibliographystyle{model5-names}
% \bibliographystyle{model6-num-names}

\bibliography{sample}


\end{document}

%%
%% End of file `elsarticle-template-num.tex'.
